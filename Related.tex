\section{Related Works}
With the development of mobile technologies and computations, obtaining and processing large-scale spatio-temporal mobility data in real world is available.Data derived from cellular networks and geo-tagged social medias have aroused great interests for people to identify and understand cities' dynamics $[1,2,3]$ comprehensively. There are mainly two ways to collect datasets to investigate human mobility $[4]$: $(1)$ data collected directly from mobile devices and $(2)$ traces collected by mobile operators. To collect data from mobile devices will be limited by two cases: the limited number of users sampled and the limited geo-tagged mobile applications, which can degrade the performance of human mobility analysis tools. Data collected by mobile operators are consecutive when the mobile devices are connected, which enables the researches of the whole network behaviors $[5]$. Several analysis tools have been proposed to help us understand the human mobility patterns. Becker et al. $[1]$ analyzed people flow in and out a suburban city as well as identify usage patterns based on call detail records
$(CDRs)$. Yuan J et al. $[6]$ demonstrated the existence of different functions regions in a city through GPS trajectory datasets. Reades J et al. $[7]$ connected telecoms usage data to a geography of human activity to analyze cities. In this paper, we try to understand the human mobility of large-scale cellular towers through data collected by an Internet Service Provider $(ISP)$.

Lots of researchers focused their attentions on the community structure-detection works, such as the detections of social networks, biochemical networks, and so on $[8,9]$, where varieties of community detection methods have been proposed to analyze the statistical characteristics of different network systems. In $[9]$, community structure is analyzed by Girvan and Newman (GN) edge betweenness method, which iteratively measures the betweenness for all edges in the networks and removes the edges with the highest betweenness until no edges remained. GN performance will be limited by low implementation speed and high computational requirements, with running-time $O(n^3)$ on sparse graph, where $n$ denotes the number of vertices. Therefore, GN is only suitable for small networks. M. E. J. Newman. et al. $[10]$ proposed a fast greedy algorithm based on the modularity. Compared with GN, the fast greedy algorithm avoids the operations of removing edges and pruning maximum distance iteratively, which makes the method achieve a high-speed advantage and run in time $O(n^2)$ on a sparse graph. However, it is impractical for small networks $[11]$. Pons, P and Latapy, M $[12]$ proposed a random walk method, called \emph{Walktrap}, to measure the similarities between vertices. This algorithm performs well in capturing the community structure and computation efficiency, with runing-time $O(n^2logn)$ in most practical cases, where n is the number of vertices in the input graph. However, occupying too much memory makes Walktrap impractical for networks with large graphs. Map equation based on flows yielded by networks is considered by M. RosvallC and D. Axelsson $[13]$ to highlight and simplify networks structure. Compared with fast greedy, map equation is tended to work on the patterns of network flow. The performances of all community structure-detection methods mentioned above are also illustrated in this paper.

The map equation based on information theory performs well in simplifying and highlighting essential regularities of network flows, and can identify community modules effectively. Persson, C. et al. $[14]$ used map equation to map citation flows between scientific journals, which resulted in an efficient and accurate method to identify scientific fields overlapped in multidisciplinary journals. Edler, D. et al. $[15]$ presented Infomap Bioregions \emph{biogeographical regions}, which transforms species-distribution data into bioregion maps perfectly, which proves that there is a good match between data-driven identification of bioregions delimited by Infomap Bioregions and commonly used bioregionalization maps. Infomap method is also considered in this paper to detect the structure of a city with spatio-temporal data from cellular networks.



\section{Conclusion}